\chapter*{Introdução}
\label{ch:intro}
Atualmente a informática esta ligada a todos os ramos de atuação do homem e
incondicionalmente ligada ao cotidiano das pessoas.

Desprezar ou não levar em conta seu potencial educacional é prejudicial à
formação dos alunos atualmente.

Dessa forma, neste trabalho são propostos a criação de um laboratório de
informática de baixo custo, que permita ao corpo doscente e discente fazer
melhor uso dessa ferramenta no processo de ensino-aprendizagem. Além disso
será apresentado um software para auxiliar as disciplinas de matemática e
saúde (tema transversal) por meio do cálculo do índice de massa corporal.

Para atingir estes objetivos, serão discutidos a seguir alguns aspectos da
relação entre informática e educação e alguns casos de sucesso no emprego
da informática nas escolas.
