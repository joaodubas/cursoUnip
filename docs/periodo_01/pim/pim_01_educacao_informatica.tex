\chapter[Educação versus Informática]{Relação entre Educação e Informática}
\label{ch:edu_x_info}
A Informática vem adquirindo cada vez mais relevância no cenário educacional
\cite{lopes:2002} e, enquanto disciplina, a Informática tem como objetivos
introduzir os alunos aos principais sistemas operacionais e aplicativos
computacionais usados no meio científico-acadêmico, bem como fornecer conceitos
básicos no desenvolvimento de programas \cite{fonseca_et_al:2005}.

Porém, vivemos em um mundo tecnológico, onde a Informática é uma das
peças principais e, conceber a Informática apenas como disciplina, é ignorar
sua atuação em nossas vidas \cite{lopes:2002}. A Informática deve ser levada para toda escola
ao invés de ser restringida a uma sala, presa em um horário fixo e sob a
responsabilidade de um único professor, pois isso limita o processo de
desenvolvimento da escola como um todo e se perde assim a oportunidade de
fortalecer o processo pedagógico \cite{lopes:2002}.

A Informática então, também pode e deve ser utilizada como instrumento de
aprendizagem e, nesse sentido, a educação vem passando por mudanças
estruturais e funcionais frente a essa nova tecnologia \cite{lopes:2002}.

Houve época em que era necessário justificar a introdução da Informática na
escola. Hoje, já existe consenso quanto à sua importância e, o principal
objetivo defendido hoje ao adaptar a Informática ao currículo escolar, está
na utilização do computador como instrumento de apoio às matérias e aos
conteúdos lecionados, além da função de preparar os alunos para uma
sociedade informatizada \cite{souza:2001}.

No começo, quando as escolas começaram a introduzir a Informática no ensino,
percebeu-se, pela pouca experiência com essa tecnologia, um processo um
pouco caótico. Muitas escolas introduziram em seu currículo o ensino da
Informática com o pretexto da modernidade. Mas não identificavam exatamente
o que fazer nessa aula e até mesmo quem poderia dar essas aulas. A
princípio, contrataram técnicos que tinham como missão ensinar a Informática
Básica apenas, com quase nenhum vínculo com as disciplinas, cujos objetivos
principais eram o contato com a nova tecnologia e oferecer a formação
tecnológica necessária para o futuro profissional na sociedade \cite{fonseca_et_al:2005}.

Porém, com o passar do tempo, algumas escolas, percebendo o potencial dessa
ferramenta, introduziram a Informática Educativa, que, além de promover o
contato com o computador, tem como objetivo a utilização dessa ferramenta
como instrumento de apoio às matérias e aos conteúdos lecionados. Esse apoio
é vinculado a uma disciplina de Informática, que tem a função de oferecer os
recursos necessários para que os alunos apresentem o conteúdo das outras
disciplinas \cite{fonseca_et_al:2005}.

Para compreendermos melhor as diferentes formas como o computador tem
chegado à escola e qual delas melhor caracteriza a Informática Educativa,
\citeonline{souza:2001} faz uma classificação sobre a iniciação do computador na
escola e seus diferentes usos: Informática Aplicada à Educação, Informática
na Educação, Informática Educacional e Informática Educativa. Vejamos a
seguir em que aspectos elas se diferenciam:

\begin{itemize}

    \item A Informática Aplicada à Educação caracteriza-se pelo uso do
    computador em trabalhos burocráticos da escola, como, por exemplo,
    controle de matrículas, de notas, folhas de pagamento, tabelas,
    digitação de ofícios, relatórios e outros documentos internos da escola.

    \item A Informática na Educação corresponde ao uso do computador através
    de softwares de apoio e suporte à educação como tutoriais, livros
    multimídias, buscas na Internet e o uso de outros aplicativos em geral.
    Nesse caso, geralmente o aluno vai ao laboratório para aulas de reforço
    ou para praticar atividades de Informática Básica que, na maioria das
    vezes, não apresentam nenhum vínculo com os conhecimentos trabalhados em
    sala de aula.

    \item A informática Educacional indica o uso do computador como
    ferramenta auxiliar na resolução de problemas. Nesse caso, as atividades
    desenvolvidas no laboratório são resultantes ou interligadas a projetos.
    Os alunos podem fazer uso dos recursos informáticos disponíveis. Aqui,
    eles executam as atividades, trabalhando sozinhos no computador ou com
    auxilio de um professor ou monitor de informática. Assim, por mais bem
    planejadas que sejam as atividades geradas pelos projetos, a
    aprendizagem dos conteúdos acaba não se processando de maneira ideal,
    pois não há intervenções do professor especialista (Português,
    Matemática, etc.) para conduzir a aprendizagem.

    \item A Informática Educativa se caracteriza pelo uso pleno da
    Informática como um instrumento a mais para o professor utilizar em suas
    aulas. Aqui, o professor especialista deve utilizar os recursos
    informáticos disponíveis, explorando as potencialidades oferecidas pelo
    computador e pelos softwares, aproveitando o máximo possível suas
    capacidades para simular, praticar ou evidenciar situações, geralmente
    de impossível apreensão desta maneira por outras mídias. Nesse modelo, a
    Informática exerce o papel de agente colaborador e meio didático na
    propagação do conhecimento, posta à disposição da educação, através do
    qual o professor interage com seus alunos na construção do conhecimento
    objetivado.

\end{itemize}

De acordo com a classificação acima, podemos perceber que a Informática
poderá ser utilizada na escola através de concepções diferentes em vários
tipos de atividades. Entendemos que as quatro concepções são importantes e
não se contradizem, mais se complementam e, que, a Informática Educativa
deve ser considerada como prioritária entre as quatro modalidades, pois, só
através dela, poderão ocorrer as contribuições mais significativas para a
construção do conhecimento por parte dos alunos.

A Informática deve habilitar e dar oportunidade ao aluno de adquirir
novos conhecimentos, facilitar o processo ensino/aprendizagem, ser um
complemento de conteúdos curriculares visando o desenvolvimento integral do
indivíduo \cite{lopes:2002}.

O acesso à Informática deve ser visto como um direito e, portanto, nas
escolas públicas e particulares o estudante deve poder usufruir de uma
educação que no momento atual inclua, no mínimo, uma alfabetização
tecnológica. Tal alfabetização deve ser vista não como um curso de
Informática, mas sim como um aprender a ler essa nova mídia. Assim, o
computador deve estar inserido em atividades essenciais, tais como aprender
a ler, escrever, compreender textos, entender gráficos, contar, desenvolver
noções espaciais, etc. E, nesse sentido, a Informática na escola passa a ser
parte da resposta a questões ligadas à cidadania.
