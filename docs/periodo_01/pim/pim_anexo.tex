\appendix

\chapter{Reuniões de grupo}

\begin{description}

    \item[02 de Março] Definição do grupo e composição do tema.

    \item[09 de Março] Definição dos tópicos a serem discutidos, proposta
    inicial de trabalho e montagem do cronograma.

    \item[16 de Março] Adequação dos tópicos a serem discutidos, organização
    do cronograma.

    \item[23 de Março] Definição dos requisitos para programa a ser
    apresentado.

    \item[30 de Março] Inicio do trabalho de revisão de bibliografia.

    \item[06 de Abril] Inicio da montagem para proposta do laboratório de
    informática.

    \item[13 de Abril] Levantamento de necessidades do colégio \emph{Faz de
    contas}.

    \item[20 de Abril] Inicio da organização das referências revisadas e
    escrita do texto provisório.

    \item[27 de Abril] Revisão dos conteúdos já montados e progresso do
    trabalho.

    \item[04 de Maio] Formatação do texto relativo ao capítulo 2.

    \item[11 de Maio] Divisão do trabalho para o novo integrante do grupo e
    revisão do progresso.

    \item[18 de Maio] Revisão do programa a ser montado, definição de novos
    requisitos e revisão do que já foi montado.

\end{description}

\chapter{Requisitos do software}

    \section{Propósito}
    O software deverá seguir as seguintes orientações:

    \begin{enumerate}

        \item Permitir cadastrar uma turma

            \begin{itemize}

                \item Nome da turma

                \item Ciclo da turma

                \item Período da turma \{1: `matutino', 2: `vespertino',
                    3: `noturno', 4: `integral'\}

            \end{itemize}

        \item Permitir cadastrar uma avaliação de uma turma

            \begin{itemize}

                \item Quantidade de alunos na turma

            \end{itemize}

        \item Permitir cadastrar diversas avaliações de alunos

            \begin{itemize}

                \item Quantidade de avaliações do aluno

            \end{itemize}

        \item A respeito do aluno serão cadastradas

            \begin{itemize}

                \item Nome do aluno

                \item Gênero do aluno \{1: `masculino', 2: `feminino'\}

                \item Data de nascimento

            \end{itemize}

        \item A respeito da avaliação serão cadastradas

            \begin{itemize}

                \item Data da avaliação

                \item Massa corporal (kg)

                \item Estatura (cm)

                \item Circunferência da cintura (cm)

                \item Circunferência do quadril (cm)

            \end{itemize}

    \end{enumerate}

    Ao professor será emitido um laudo com:

    \begin{itemize}

        \item Idade de cada aluno (em anos e em meses)

        \item Índice de massa corporal (IMC) de cada aluno

        \item Classificação em percentil do IMC pela idade

        \item Relação cintura/quadril (RCQ) de cada aluno

        \item Índice de conicidade (IC) de cada aluno

        \item Classificação da RCQ e IC de cada aluno

        \item Representação gráfica da massa corporal, estatura e IMC
        pela idade (caso tenhamos um protótipo)

    \end{itemize}

    No trabalho é interessante mostrar a ligação entre os assuntos
    envolvidos no desenvolvimento do aplicativo e as disciplinas que
    gostaríamos de envolver no processo de ensino (matemática, educação
    física e saúde).

    \section{Cálculos}

        \subsection{Índice de massa corporal (IMC)}
        O IMC ($kg/m^2$) é calculado de acordo com a equação:

        \begin{equation}
        \label{eqn:imc}
        IMC=\frac{MC}{H^2}
        \end{equation}

        Onde: $MC$ é a massa corporal em quilogramas e $H$ é a
        estatura em metros.

        \subsection{Relação cintura quadril (RCQ)}
        A RCQ é calculada de acordo com a equação:

        \begin{equation}
        \label{eqn:rcq}
        RCQ=\frac{CCT}{CQD}
        \end{equation}

        Onde: $CCT$ é a circunferência de cintura em centímetros e $CQD$
        é a circunferência de quadril em centímetros.

        \subsection{Índice de conicidade (IC)}
        O IC é calculado de acordo com a equação:

        \begin{equation}
        \label{eqn:ic}
        IC=\frac{CAB}{(0,109\times\sqrt{\frac{MC}{H}})}
        \end{equation}
