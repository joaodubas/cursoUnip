\documentclass[a4paper]{report}
\usepackage[brazil]{babel}
\usepackage[utf8]{inputenc}

\begin{document}

    \title{Portfólio\\Atividades Complementares}
    \author{João Paulo Dubas\\RA: A54467-4}
    \date{Santos - Praia\\2010}

    \maketitle

    \tableofcontents

    \chapter[Webdesign versus Javascript]{Palestra Impacta: Webdesign versus
    Javascript}

        \section{Introdução}
        O objetivo desta palestra foi apresentar as distinções entre as
        atribuições do \emph{webdesigner} e do \emph{programador} numa
        equipe de desenvolvimento web, bem como, no mercado atual estas
        duas funções acabam se unindo de maneira inadequada.

        Além disso, fez-se a introdução de técnicas modernas de
        desenvolvimento javascript, e como está linguagem pode ajudar
        o desenvolvedor a criar aplicativos web (\emph{web applications})
        que funcionem de forma similar em todos os navegadores com o
        uso de \emph{frameworks}, tais como, jQuery, Mootools, Pryx
        entre outros.

        \section[Descrição atividade]{Descrição da atividade
        \\em ordem cronológica}
        Foi apresentado um histórico bastante rigoroso a respeito da
        evolução do processo de desenvovlvimento na internet, desde o
        desenvolvimento da linguagem \emph{javascript}, passando pela
        especifcação do \emph{CSS}, até os dias atuais, onde a
        separação entre os \emph{web applications} e os aplicativos
        desktop é cada vez mais tênue.

        Definiu-se os papéis ideais do \emph{webdesiger} e do
        \emph{webdeveloper}, conforme apresentado abaixo:

        \begin{description}

            \item[\emph{Webdesigner}] Deve ter como principal meta a de planejar
            a interface com o usuário, a experiência usuário x
            interface, como as informações devem ser dispostas na
            página e, em doses homeopáticas, pode alterar o código
            fonte do aplicativo.

            \item[\emph{Webdeveloper}] Tem como objetivo criar os mecanismos para
            que as interações previstas pelo \emph{webdesigner} tornem-se
            realidade. Para tal utiliza de ferramentas, tais como o javascript,
            para realizar suas tarefas.

        \end{description}

        é importante notar, que não foram discutidos os papéis do
        administrador de sistema, bem como, o desenvolvedor do \emph{backend}.

        Apesar do quadro ideal ter sido apresentado anteriormente, o pan\^{o}rama
        atual mostra, que ambos os profissionais, acabam se mesclando e
        realizando tarefas para as quais não se prepararam, como por exemplo,
        o \emph{webdesigner} também trabalha no código fonte, equanto o
        \emph{webdeveloper} se vê obrigado a criar a iterface com o
        usuário.

        Feita a relação entre o \textbf{ideal} e o \textbf{atual},
        discutiu-se uma proposta de \emph{workflow} para otimizar o trabalho do
        profissional que trabalha com internet. Este fluxo é apresentado
        abaixo:

        \begin{enumerate}

            \item Defina o foco do trabalho a ser produzido.

            \item Escolha as ferramentas para realizar o trabalho.
                \\Uma bom ambiente integrado de desenvolvimento (\emph{IDE})
                faz toda a diferença.

            \item Crie os \emph{wireframes} e \emph{protótipos} para o
            projeto em trabalho.

            \item Converta os protótipos em algo que o cliente possa
            interagir.

            \item Escolha os frameworks para auxiliá-lo na tarefa de
            desenvolver o trabalho final.

        \end{enumerate}

        Com o \emph{workflow} discutido, apresentou-se uma das \emph{IDE}s da
        Adobe, o \emph{Dreamweaver} em sua versã CS5. Um programa excepcionaç
        para o desenvolvedor web\footnote{Apesar de não utilizá-lo no meu
        dia a dia, prefiro as soluções \emph{open-source} como o Vim.}.

        Enfim, discutiram-se alguns dos \emph{frameworks} javascript mais
        utilizados na atualidade, e foram apresentados alguns usos para destes.

        \section{Conclusão}

        A palestra foi bastante interessante, não tão aprofundada quanto
        eu esperava, mas serviu para mostrar como o mercado de desenvovlvimento
        web, em nosso país ainda é imatura frente a outros, além
        disso, foram apresentadas algumas boas práticas de programação.

    \chapter[15\textordmasculine EDTED]{15\textordmasculine Encontro de Design e
    Tecnologia Digital}

        \section{Introdução}
        O encontro teve como cenário apresentar tecnologias emergentes,
        tendências de mercado e prover informaçoes práticas para
        seus participantes.

        Os conteúdos foram dividos em palestras, ministradas por profissionais
        renomados, dentro de sua área de atuação. Sendo assim, não
        foi possível atender a todos os temas.

        \section[Descrição atividade]{Descrição da atividade
        \\em ordem cronológica}
        
            \subsection[\emph{Product Backlog}]{\emph{Product Backlog}: elaboração e
            \\manutenção com garantia de ROI}
            A discussão da garantia de retorno do investimento (ROI) por
            meio da definição das atividades a serem realizadas de
            acordo com suas prioridades e como o \emph{Product Owner} e o
            \emph{Scrum Master} devem trabalhar para garantir que as atividades
            definidas no backlog, foi o tópico central desta palestra.

            Após definir os \emph{rituais} pertinentes ao modelo \emph{Agile},
            quais são os papéis desenvolvidos por cada atuante (\emph{Product Owner},
            \emph{Scrum Master} e o \emph{Team}, foram apresentadas técnicas
            para que o Scrum Master melhore sua relação com o Product Owner
            e dê a este a confiança necessária para expor ao time quais são
            as necessidades do produto e as prioridades a serem desenvolvidas
            para que o produto de retorno do investimento realizado para que
            este se torne um caso de sucesso.

            \subsection{Ruby on Rails}
            Nesta palestra, foi apresentada uma visão geral a respeito da
            comunidade, ou melhor, do ecossistema que se formou ao redor do
            framework \emph{Ruby on Rails}.

            \subsection{Games}

            \subsection{Acessibilidade na web}

            \subsection{\emph{Start-ups}: André Felipe - Design Atento}

            \subsection{\emph{Mobile tagging} \& Realidades mistas}

        \section{Conclusão}

    \chapter[Python básico]{Python básico para Django e Google App Engine}

        \section{Introdução}

        \section[Descrição atividade]{Descrição da atividade
        \\em ordem cronológica}

        \section{Conclusão}

    \chapter[Django ORM]{Django ORM: SQL sem sujar as mãos}

        \section{Introdução}

        \section[Descrição atividade]{Descrição da atividade
        \\em ordem cronológica}

        \section{Conclusão}
\end{document}
